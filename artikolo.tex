\documentclass[utf8]{scrartcl}

\usepackage{fontspec}

\usepackage{polyglossia}
\setdefaultlanguage{esperanto}
\usepackage{lmodern}
\usepackage{tikz}
\usetikzlibrary{automata, positioning, arrows, calc}
\usepgflibrary{fpu}
\pgfdeclarelayer{bg}    % declare background layer
\pgfsetlayers{bg,main}

\usepackage{amsfonts}
\usepackage{amsmath}

\DeclareMathOperator{\pgkd}{pgkd}


\title{Postkvantuma ĉifrado}
\author{Johannes Mueller}


\begin{document}

\section{Enkonduko}

Antaŭ ol ni rigardas la temon de postkvantuma ĉifrado, ni konsideru la signifon
de ĉifrado ĝenerale.  Ĝi apartenas al la pli vasta temaro de datuma sekureco
kaj datuma protektado.  Oni okupiĝas pri tiu temaro ĉefe celante la tri jenajn
celojn:
%
\begin{description}
\item[konfidenteco] neniu necelata povas legi la datumojn
\item[fidindeco] pruveblo de la deveno kaj la aŭtenteco
\item[disponeblo] disponebla kiam bezonataj
\end{description}

Ĉiuj tri de tiuj celoj estas gravaj por la funkciado de la nuntempa socio.  Ni
ĉiuj fidas je tio, ke niaj komunikaĵoj privataj kaj ankaŭ ekzemple la
sekretaĵoj kiel pasvortoj ne estu alireblaj por krimuloj – konfidenceco. Same
gravas ke neniu krimulo povu falsi ekzemple nian komunikon kun la bankoj –
fidindeco.  Kaj trie, gravas ke signifaj datumoj estu je dispono kiam
bezonata. Se ne tutaj malsanulejoj, flughavenoj aŭ fevojsistemoj ĉesas funkcii.

Por du el tiuj tri datumprotektaj celoj, la konfidenceco kaj la fidindeco,
ĉifrado ludas ĉefan rolon.

\subsection{Terminologio}

Komence ni enkonduku kelkajn terminojn.
%
\begin{description}
\item[klarmesaĝo] la neĉifrita mesaĝo, legebla por ĉiuj
\item[ĉifraĵo] la ĉifrita mesaĝo, legebla nur por tiuj kiuj havas la ŝlosilon
\item[ĉifri] transformi klarmesaĝon al ĉifraĵo per ŝlosilo
\item[malĉifri] transformi ĉifraĵon al klarmesaĝo per ŝlosilo
\item[ŝlosilo] datumo por ĉifri aŭ malĉifri mesaĝon
\item[haketo] testsumo por pruvi ĉu mesaĝo estas aŭtenta
\item[subskribo] pruvilo ke mesaĝo kaj la deveno estas aŭtentaj
\end{description}



\section{Klasika ĉifrado}

La bazaj principoj de la moderna ĉifrado estis vortumitaj une de Kerckoff:
%
\begin{itemize}
\item La sistemo ne estu sekreta (nur la ŝlosilo). Oni devas supozi ke la
  atakanto scias kiel niaj ĉifrosistemoj funkcias. Krome ili estu publikaj, por
  ke spertuloj povu esplori ilin por taksi la sekurecon.
\item La ĉifraĵo ne estu distingebla de hazarda datumo.
\item La ĉifraĵo estu transsendebla elektronike.
\item La sistemo estu transportebla kaj uzebla pere de unu persono.
\item La sistemo estu facile uzebla.
\end{itemize}


\subsection{Simetria ĉifrado}

La simetria ĉifrado estas la pli facile komprenebla. Oni malĉifras la ĉifraĵon
per la sama ŝlosilo per kiu oni ĉifris la klarmesaĝon.  Tio estas ekzemple tre
taŭga por ĉifri la datumoj sur durdiskoj aŭ similaj datumujoj.  Kiam oni laĉas
la komputilon aŭ aligas datumujon, oni devas ielmaniere enigi la ŝlosilon,
ekzemple per tajpante ĝin. Poste la komputilo aŭtomate ĉifras ĉiujn skribaĵojn
al la disko kaj malĉifras ĉiuj legaĵojn.  Simetriaj ĉifrosistemoj estas tre
rapidaj, kaj devas esti tre rapidaj, ĉar malrapida aliro al durdisko bremsus la
tutan sperton de komputila uzado.

La granda malavantaĝo de simetria ĉifrado estas, ke la du komunikpartneroj
devas interŝanĝi la ŝlosilon. Tio estas malfacila, ĉar por tio ja ankoraŭ ne
ekzistas sekura komunikkanalo. Atakanto povus intercepti la ŝlosilon antaŭ la
unua vera mesaĝo estas trnassendata.

\begin{figure}
  \begin{tikzpicture}[
    ->,>=stealth,shorten >=1pt,
    auto,
    node distance=5cm,
    ]
    \node [fill=green!33] (km) {Klarmesaĝo $K$};
    \node [fill=red!33, right of=km] (cf) {Ĉifraĵo $M$};

    \draw (km) edge[bend left] node {$f_C(K, S)$} (cf);
    \draw (cf) edge[bend left] node {$f_M(M, S)$} (km);
  \end{tikzpicture}
  \caption{La principo de simetria ĉifrado.}
\end{figure}

\subsection{Malsimetria ĉifrado}

Malsimetria ĉifrado solvas la problemon de la interŝanĝo de la ŝlosiloj. Oni
havas unu ŝlosilon por ĉifri la mesaĝon – la publikan ŝlosilon. Por malĉifri la
mesaĝon oni bezonas la privatan ŝlosilon.  Por partopreni la komunikadon oni
unue devas krei tian ŝlosilparon kaj doni la publikan ŝlosilon al la
komunikpartnero.  Same oni riecevas de la komunikpartnero ties publikan
ŝlosilon por ĉifri mesaĝojn.

La malavantaĝo de malsimetria ĉifrado estas, ke ĝi estas multi pli malrapida aŭ
bezonas pli da komputila potenco.  En la praktiko oni uzas kombinon de ambaŭ
sistemoj. Oni kreas hazardan simetrian ŝlosilon. Per ĝi oni ĉifras la mesaĝon
kaj poste ĉifras nur la simetrian ŝlosilon per la publika ŝlosilo de la
komunikpartnero.  Tiu poste uzas sian privatan ŝlosilon por malĉifri la
simetrian ŝlosilon kaj per ĝi malĉifras la mesaĝon.
%
\begin{figure}
  % \begin{tikzpicture}[node distance=0.5em,align=left]

  %   \node (anjo) at (0,0) {Anjo};
  %   \node (boĉjo) at (6,0) {Boĉjo};
  %   \node (publika) at ($(anjo)!0.5!(boĉjo)$) {publiko};

  %   \node [below=of anjo] (priv) {$S_{priv}$}};
  %   \node [below=of priv] (pub_a) {$S_{pub} = P(S_{priv})$}};
  %   \node (pub_b) at (pub_a -| boĉjo) {{S_{pub}$}};
  %   \node (pub) at ($(pub_a)!0.5!(pub_b)$) {{S_{pub}$}};

  %   \draw[->] (pub_a) -- (pub);

  %   \draw[->] (pub) -- (pub_b);


  %   \node [below=of pub_b] (klar) {$K$};

  %   \node [below=of klar] (cf_b) {$M = f_C(K, S_{pub})$};

  %   \node (cf_a) at (cf_b -| anjo) {$M$};
  %   \draw[->] (cf_b) -- (cf_a);

  %   \node [below=of cf_a] (mcf) {$K = f_M(M, S_{priv})$};

  %   \begin{pgfonlayer}{bg}
  %     \fill [fill=red!20]
  %     ($(anjo) + (-4em,2em)$) rectangle ($(mcf) + (4em, -2em)$) coordinate (anjo border);

  %     \fill [fill=green!20]
  %     (anjo border) rectangle ($(boĉjo) + (-4em, 2em)$) coordinate (boĉjo border);

  %     \fill [fill=magenta!20]
  %     ($(boĉjo) + (4em, 2em)$) rectangle (anjo border -| boĉjo border);
  %   \end{pgfonlayer}

  % \end{tikzpicture}
  \caption{La principo de malsimetria ĉifrado}
\end{figure}

Estas grave emfazi, ke la tiaj ĉifrosistemo estas io tute ĉiutaga. Ĉiu
retumilo, ĉiu retpoŝta programo, multaj tujmesaĝilo (Telegramo kutime ne)
aplikas tiaspecan ĉifradon aŭtomate.


\section{Matematikaj princpoj de ĉifrado}

En la antaŭa ĉapitro ni konstatis, ke por simetria ĉifrado oni devas krei
ŝlosilan paron de privata kaj publika ŝlosilo.  Gravas, certe, ke la privata
ŝlosilo ne estu kalkulebla de la publika. Ni povas imagi, ke ni elektas
hazardan privatan ŝlosilon kaj el ĝi kalkulas la privatan. Ni do bezonas kion
oni nomas \emph{unudirektan funkcion}.  Tio estas matematika funkcio, kiu estas
facile kalkulebla, sed multe malplifacile inversebla.

Ekzemplo por tia unudirekta funkcio estas la produkto de du primoj.  Kalkuli la
produkton de du primoj estas facila. Estas unu multobliga operacio.  La
inverso, la faktorigo – havante la produkton de du primoj trovi la du primajn
faktorojn estas multi malpli facila, ĉar oni fakte devas elprovi ĉiujn eblojn.
Plie la komplekseco – la bezonata laboro por la faktorigo – kreskas pli rapide
ol tiu por la multobligo.

Po havi ian impreson pri la ciferoj jen ekzemplo. En la jaro 2020 oni starigis
la rekordon por faktorigo. Oni faktorigis 250-ciferan, t.s. 829bitan nombron en
2700 procesilaj horoj Intel Xeon Gold 6130 je 2.1 GHz.  Por kutimaj
ĉifrosistemoj oni uzas 2048 aŭ eĉ 4096bitajn nombrojn.

La interesa demando pri unudirektaj funkcioj estas, ĉu oni fakte estas certa,
ke la inversa direkto estas tiom multe pli malfacila. Fakte oni nur, ĉär oni
dum jardekoj de klopodoj ne sukcesis trovi pli facilan metodon.  Almenaŭ
funkciantan sur klasikaj komputiloj.


\section{Minaco de kvantuma komputilado}

La supozo ke la faktorigo de primprodkto estas malfacila sur klasika komputilo
ĝis nun staras.  Sed en la 1990aj jaroj oni teorie inventis la kvantuman
komputilon.  Kvankam ĝi tiatempe estis nur matematika modelo, Peter Shor
inventis por ĝi algoritmon por faktorigi primproduktojn kun multe malpli alta
komplekseco.  Por kompreni la algoritmon de Shor, ni unue devas kompreni la
principojn de kvantuma komputilo.


\subsection{Principoj de kvantuma komputilo}

Klasika komputilo ĉiam havas difinitan enigon, difinitan operacion kaj tial
ankaŭ difinitan eligon, t.e. rezulton.  La stato de la komputilo ĉiam estas difinita.

\begin{figure}
  \begin{tikzpicture}[>=stealth,shorten >=1pt]
    \begin{scope}[every node/.style=draw,inner xsep=15pt]
      \node [draw] (input) {$5$};
      \node [draw, right=3em of input] (operation) {$+3$};
      \node [draw, right=3em of operation] (output) {$8$};
    \end{scope}
    \node [above=0.1em of input] {enigo};
    \node [above=0.1em of operation] {operacio};
    \node [above=0.1em of output] {eligo};

    \draw [->] (input) -- (operation);
    \draw [->] (operation) -- (output);
  \end{tikzpicture}
  \caption{Klasika komputilo havas difinitan staton.  La eniga registro havas
    staton \texttt{3}, la eliga registro havas staton \texttt{5}.}
\end{figure}

Kontraste al klasika komputilo, kvantuma komputilo ne estas en difinita stato
sed en \emph{superpozicio} de pluraj statoj samtempe.  La enigo estas ne unu
difinita nombro, sed superpozicio de pluraj. La eligo etas la superpozicio de
ĉiuj enigoj kun la operacio aplikita.

\begin{figure}
  \begin{tikzpicture}[>=stealth,shorten >=1pt]
    \begin{scope}[every node/.style=draw]
      \node [draw] (input) {
        $\left|\textcolor{red}{1}\right> + \left|\textcolor{blue}{2}\right> +
        \left|\textcolor{magenta}{3}\right>
        + \left|\textcolor{green}{4}\right> + \left|\textcolor{cyan}{5}\right> $
      };
      \node [draw, right=1em of input] (operation) {$+3$};
      \node [draw, right=1em of operation] (output) {
        $
        \left|\textcolor{red}{4}\right> + \left|\textcolor{blue}{5}\right> +
        \left|\textcolor{magenta}{6}\right>
        + \left|\textcolor{green}{7}\right> + \left|\textcolor{cyan}{8}\right>
        $};
    \end{scope}
    \node [above=0.1em of input] {enigo};
    \node [above=0.1em of operation] {operacio};
    \node [above=0.1em of output] {eligo};

    \draw [->] (input) -- (operation);
    \draw [->] (operation) -- (output);
  \end{tikzpicture}
  \caption{La eniga registro de kvantuma komputilo estas superpozicio de pluraj
    statoj, same la eliga registro.}
\end{figure}

Superpozicio signifas ke la registro estas kvazaŭ samtempe en pluraj statoj.
Kiam oni kontrolas la staton de la registro, la superpozicio kolapsas kaj
restas hazarde unu el le statoj de la superpozicio.

Ni ja lernis, ke por faktorigo de du primoj oni konas nur la metodon, elprovi
ĉiujn eblojn.  Do kio se oni metas superpozicion de ĉiuj ebloj al la eniga
registro de kvantuma komputilo?  La eliga registro tiam estas superpozicioj de
ĉiuj produktoj, sed tio ne helpas, ĉar se oni kontrolas la rezulton, la
superpozicio kolapsas al unu el la statoj kaj oni ne gajnis ion ajn.


\subsection{La algoritmo de Shor}

Meze de la 1990aj jaroj \textsc{Peter Shor} inventis algoritmon, kiu uzante
kvantuman komputilon kapablas faktorigi du nombrojn en logaritma
komplekseco. Ni nun traludas ĝin matematike.

\minisec Tasko: faktorigi la nombron

\begin{equation}
  \label{eq:produkto}
  N = a b \;\text{kun}\; a, b \in \mathbb{N}
\end{equation}
%
Ni divenas $d$:
\begin{equation}
  \label{eq:diveno}
  1 < d < N
\end{equation}
%
Ni kalkulas la plej grandan komunan divizoron de $d$ kaj $n$.
\begin{equation}
  \label{eq:diveno-sukceso}
  \text{se} \pgkd(d, N) > 1 \Longrightarrow\text{sukceso}
\end{equation}
%
Ni scias ke ekzistas numero $p$:
\begin{equation}
  \label{eq:ekzistas-p}
  \exists p  \in \mathbb{N} \left|  \right. d^p = m N + 1 \;\text{kun}\; m \in \mathbb{N}
\end{equation}
%
Iomete rearanĝinte la antaŭan ekvacion ni ricevas:
\begin{equation}
  \label{eq:dp-divizoroj}
  d^p - 1 = \left(d^{p/2} + 1\right)\left(d^{p/2} - 1\right) = m N
\end{equation}
%
Oni povas kalkulo ke en $37,5 \%$ de la divenoj
\begin{itemize}
\item $p$ estas para
\item $d^{p/2} \pm 1$ estas divizoro de $N$
\end{itemize}

La tasko do, estas trovi la $p$ de nia diveno $d$. Se ni tiam estas inter tiuj
$37,5 \%$ ni sukcesis, se ne, ni ripetas kaj post kelkaj provoj kun granda
probableco trovis divizoron de $p$.

Sekvas nun la kvantuma parto de la algoritmo de Shor. Unue kelkaj ekvacioj. Por
ĉiuj niaj divenoj $d$ ni povas trovi por ĉiu $i\in\mathbb{N}$ du naturajn
numberojn $m_i$ kaj $r_i$ tiel ke:
\begin{equation}
  \label{eq:di-rilato}
  d^i = m_i N + r_i
\end{equation}
%
Se ni tiun ekvacion multobligas kun \eqref{eq:ekzistas-p} ni ricevas:
\begin{equation}
  \label{eq:di-p-rilato}
  d^{i+p} = m_{i+p} N + r_i
\end{equation}
%
Ĉar $d$, $m$, kaj $N$ estas naturaj nombroj, $r$ estas la resto (la modulo) de
la divido $d_i/N$
\begin{equation}
  \label{eq:di-mod-ri}
  d^{i+p}\mod N = r_i
\end{equation}

Tio estas sufiĉe interesa resulto, ĉar tio signifas, ke ni trovas la saman
$r_i$ por ĉiu $p$a $i$. Do se ni trovas $r_i$ por ia nombro $i$, ni trovas la
saman $r_i$ por $i + p$ kaj por $i + 2p$ kaj por ĉiuj $i + np$.

Nun la la kvantuma komputilo. Ni komencas per superpozicio de ĉiuj eblaj
nombroj por $p$.
\[\left|1\right> + \left|2\right> + \left|3\right> + \left|4\right> +
  \left|5\right> + \left|6\right> + \left|7\right> + \left|8\right> + \ldots \]
%
Al tiu eniga registro ni aplikas la operacion $d^x$ en kiu $d$ estas nia
diveno. Poste la stato de la kvantuma komputilo estas la jena:
\[\left|d^1\right> + \left|d^2\right> + \left|d^3\right> + \left|d^4\right> +
  \left|d^5\right> + \left|d^6\right> + \left|d^7\right> + \left|d^8\right> +
  \ldots\]
%
Nun ni aplikas la operacion $x\mod N$ kaj ricevas:
\[\left|r_1\right> + \left|r_2\right> + \left|r_3\right> + \left|r_4\right> +
  \left|r_5\right> + \left|r_6\right> + \left|r_7\right> + \left|r_8\right> +
  \ldots\]
%
Ni scias, ke la sama $r_i$ ripetiĝas ĉiujn $p$ fojojn. Do se $p=3$ aspektus
tia:
\[\left|r_1\right> + \left|r_2\right> + \left|r_3\right> + \left|r_1\right> +
  \left|r_2\right> + \left|r_3\right> + \left|r_1\right> + \left|r_2\right> +
  \ldots\]
%
Do se ni iel povas mezuri la periodon de la tiu konstanta ripetiĝo, ni havus
nian serĉatan valoron por $p$.

Por tio ekzistas fizika metodo de kvantuma Fourier transformo. Fourier
transformoj estas matematika operacio, kalkuli la frekvencan spektron de ia
signalo. Kaj la kvantuma Fourier transformo mezuras la frekvencan spektron de
kvantuma superpozicio.

Se ni do aplikas ĝin al la registro de nia kvantuma komputilo, nia kvantuma
komputilo restas en difinita stato:
\[\left|\frac{1}{p}\right>\]
%
Tiun staton ni povas mezuri kaj de tio kalkuli la valoron por $p$. Nun ni havas
la $37,5$ elcentan ŝancon ke ni jam havas niajn divizoron
\[d^p \pm 1\].

\section{Postkvantuma ĉifrado}

Ĝis nun la Shor algoritmo estas ne pli ol matematika modelo. Oni ja sukcesis
per kvantuma komputilo faktorigi la nombron $15$ sed ne pli altan nombron. Tio
signifas kaj la vojo ĝis kvantumaj komputiloj ankaŭ estas longa. Tamen neniu
scias, kiom rapida la evoluo de kvantumaj komputiloj estos. Ĉu ili endanĝerigas
nian komunikon jam post kvin jaroj? Ĉu post dek? Ĉu dudek? Aŭ eble eĉ neniam.

Pro tio nescio jam en la 2000aj jaroj ĉifrosciencistoj lanĉis la disciplinon de
\emph{postkvantuma ĉifrado}. La celo de postkvantuma ĉifrado estas elpensi
ĉifrosistemojn imunaj kontraŭ la atakoj de kvantuma komputilo. En 2017 la Usona
Nacia Instituto de Normoj kaj Teknologio (NIST) lanĉis konkurson en kiu
partoprentantoj provis aligi siajn kandidatojn por postkvantuma
ĉifrosistemo. La plej spertaj ĉifrosciencistoj de la mondo tiam klopodis trovi
malfortojn en tiuj kandidatoj. Post kvar rondoj en 2024 restis kvar algoritmoj
por kiu oni ne trovis funkciantan atakon. Unu el ili nomiĝas
\textsc{Crystals}-Kyber.


\subsection{Laticbazita ĉifrado}

Latico estas grupo de matematikaj punktoj kun la trajto, ke oni de ĉiu punkto
per taŭga kombinoj la samaj vektoroj venas al ĉiu alia punkto de la
grupo. Figuro \ref{fig:latico} montras ekzempon por du dimensia latico.  Por
$n$-dimensia latico sufiĉas ajna grupo de $n$ vektoroj, kiuj ne estas
paralelaj.

La plej simpla vektorogrupo sufiĉanta por per kombino de ajna punkto de la
latico al ajna alia nomiĝas bazo. La bazo konsistas el unu vektoro por ĉiuj
dimensio. Dudimensia latico bezonas tial du bazajn vektorojn $b_1$ kaj $b_2$.

\begin{figure}
  \begin{tikzpicture}
    \coordinate (Origin)   at (0,0);

    \coordinate (Bone) at (0.3,0.7);
    \coordinate (Btwo) at (0.7,0.1);

    \coordinate (BoneB) at ($3*(Bone)+7*(Btwo)$);
    \coordinate (BtwoB) at ($4*(Bone)+9*(Btwo)$);

    % Latice points along b1+b2

    \clip (-1, -1) rectangle (9, 4.5);

    \foreach \i in {-5, -4, -3, -2, -1, 0, 1, 2, 3, 4, 5, 6, 7, 8, 9, 10} {
      \foreach \j in {-5, -4, -3, -2, -1, 0, 1, 2, 3, 4, 5, 6, 7, 8, 9, 10, 11,
        12, 13 ,14, 15}
      \draw [fill = black]($\i*(Bone)+\j*(Btwo)$) circle (2pt);
    }
    % Draw the vectors
    \draw [ultra thick,-latex] (Origin) -- (Bone) node [above left] {$b_1$};
    \draw [ultra thick,-latex] (Origin) -- (Btwo) node [below right] {$b_2$};
    \draw[fill=gray] ($3.1*(Bone)+4.2*(Btwo)$) circle (2pt);
    \draw [thick,-latex,magenta] (Origin) -- ($3*(Bone)+7*(Btwo)$) node [above left] {$b'_1$};
    \draw [thick,-latex,magenta] (Origin) -- ($4*(Bone)+9*(Btwo)$) node [above left] {$b'_2$};
    \draw[blue] ($3*(Bone)+4*(Btwo)$) circle (3pt) node [above left] {$K$};
    % \draw [->] (Origin) -- ++ ($-11*(BoneB)+9*(BtwoB)$);
    \draw[fill=gray] ($3.1*(Bone)+4.2*(Btwo)$) circle (2pt) node [above right]
    {$M$};
  \end{tikzpicture}
  \caption{Dudimensia latico kun la bazaj vektoroj $b_1$ kaj $b_2$}
  \label{fig:latico}
\end{figure}

Figuro \ref{fig:latico} ni vidas du bazojn por la sama latico, $(b_1, b_2)$ kaj
$(b'_1, b'_2)$. Ambaŭ taŭgas kiel bazo. Tio signifas, ke oni per kombino de la
vektoroj $b_1$ kaj $b_2$ oni povas de ĉiu punkto de la latico atingi ĉiujn
aliajn de la latico. Same per kombino de $b'_1$ kaj $b'_2$. Estas evidente ke
per la $(b_1, b_2)$ estas pli facile ol per la bazo $(b'_1, b'_2)$.

\subsubsection{Serĉado de la plej proksima vektoro}

La punkto $M$ en figuro \ref{fig:latico} ne estas latica punkto. La plej
proksima latica punkto estas evidente $K$. Tion ni bone povas vidi, ĉar ĉiuj
laticaj punktoj estas markitaj en la diagramo. Sed kio se ne, se ni havus la
bonan bazon $(b_1, b_2)$ verŝajne ankaŭ estas sufiĉe facile trovi $K$ de $M$,
sed kio, se ni havas nur la malbonan bazon $(b'_1, b'_2)$? Tiam estas multe pli
malfacile. Kaj jen unudirekta funkcio.

Ĉifrasistemo povas esti ekzemple ke Anjo elpensas la du vektorojn $(b_1, b_2)$
kaj el kombinoj de ili la vektorojn $(b'_1, b'_2)$. Tiujn vektorojn ŝi sendas
al Boĉjo. Boĉjo kodigas mesaĝon por Anjo kiel kombinon de $(b'_1, b'_2)$. La
rezulto estas $K$. Tiun punkton li iomete ŝovas hazarde al $M$ kaj sendas la
koordinatojn de $M$ al Anjo. Anjo konas la bonan bazon $(b_1, b_2)$ kaj per ĝi
povas facile trovi la latican punkton $K$ kaj poste rekonstrui la mesaĝon de
Boĉjo.

Tio estas ekzemplo por simpla laticbazita ĉifrosistemo. Montriĝas ke ĝi ne
estas sekura. Oni montris sukcesajn atakojn al ĝi.


\subsection{Kyber – la normigita laticbazita ĉifrosistemo}

La normigita kaj jam uzebla laticbazita ĉifrosistemo nomiĝas
\textsc{Crystals}-Kyber}. Pincipe funkcias simile kiel ni antaŭe vidis grafike.

\end{document}

\begin{frame}
  \frametitle{Serĉado de plej proksima vektoro}
  \begin{figure}
  \begin{tikzpicture}
    \coordinate (Origin)   at (0,0);

    \coordinate (Bone) at (0.3,0.7);
    \coordinate (Btwo) at (0.7,0.1);

    \coordinate (BoneB) at ($3*(Bone)+7*(Btwo)$);
    \coordinate (BtwoB) at ($4*(Bone)+9*(Btwo)$);

    % Latice points along b1+b2

    \clip (-1, -1) rectangle (9, 4.5);


    \uncover<1-2,9->{
      \foreach \i in {-5, -4, -3, -2, -1, 0, 1, 2, 3, 4, 5, 6, 7, 8, 9, 10} {
        \foreach \j in {-5, -4, -3, -2, -1, 0, 1, 2, 3, 4, 5, 6, 7, 8, 9, 10, 11,
          12, 13 ,14, 15}
        \draw [fill = black]($\i*(Bone)+\j*(Btwo)$) circle (2pt);
      }
    }
    % Draw the vectors
    \only<1-3,8->{
      \draw [ultra thick,-latex,red] (Origin) -- (Bone) node [above left] {$b_1$};
      \draw [ultra thick,-latex,red] (Origin) -- (Btwo) node [below right] {$b_2$};
    }

    \only<2-4>{\draw[fill=blue] ($3.1*(Bone)+4.2*(Btwo)$) circle (2pt);}

    \only<4->{
       \draw [thick,-latex,magenta] (Origin) -- ($3*(Bone)+7*(Btwo)$) node [above left] {$b'_1$};
       \draw [thick,-latex,magenta] (Origin) -- ($4*(Bone)+9*(Btwo)$) node [above left] {$b'_2$};
     }

    \only<5-6,10>{\draw[blue] ($3*(Bone)+4*(Btwo)$) circle (3pt) node [above left] {\color{blue}{$K$}};}
    %\draw [->] (Origin) -- ++ ($-11*(BoneB)+9*(BtwoB)$);
    \only<6->{\draw[fill=blue] ($3.1*(Bone)+4.2*(Btwo)$) circle (2pt) node [above right] {\color{blue}{$M$}};}
  \end{tikzpicture}
  \end{figure}
  \uncover<5-6,10>{%
    \[K = (-11, 9)\]
  }

\end{frame}

\section{Kyber}

\subsection*{Modula aritmetiko}

\begin{frame}
  \frametitle{Entjera ringo}

  Modulo:
  \begin{itemize}
  \item $r = a \mod q$ signifas ke $r$ estas la resto de entjera divido $a/q$.
  \item Entjeraro modulo $q$: $\mathbb{Z}_q = \left\{0, 1, 2, \ldots, q-1\right\}$
  \item Ene de $\mathbb{Z}_q$ ĉia aritmetiko estas farataj $\mod q$.
  \end{itemize}
  \begin{columns}
    \begin{column}<3->{0.5\textwidth}
      Ekzemple por $\mathbb{Z}_{17}$:
      \begin{itemize}
      \item $9 + 15 = 7$
      \item $9 - 15 = 11$
      \item $9 \times 15 = 16$
      \end{itemize}
    \end{column}
    \begin{column}<2->{0.5\textwidth}
      \begin{tikzpicture}
        \draw (0, 0) circle (5em);
        \foreach \n in {0,1,...,16} {%
          \pgfmathparse{-360 * \n / 17 + 90}\edef\a{\pgfmathresult}
          \draw [fill] (\a:5em) circle(2pt);
          \node at (\a:4.3em) {\tiny\n};
        }
        \node at (0, 0) {$\mathbb{Z}_{17}$};
      \end{tikzpicture}
    \end{column}
  \end{columns}
\end{frame}

\begin{frame}
  \frametitle{Polinoma ringo}
  \begin{itemize}
  \item Polinomo $n$-a grada: $c_nx^n + c_{n-1}x^{n-1} + \ldots + c_1 x + c_0$
  \item $\mathbb{Z}_q[x]$: aro de ĉiuj polinomoj kun $c_n \in \mathbb{Z}_q
    \;\forall n$

  \end{itemize}

  \vspace{2em}
  \uncover<2->{Ekzemple $q=7$}
  \begin{itemize}
  \item<3-> $f(x) = 3x^3 + 4x^2 + 5$ kaj $g(x) = 2x^2 + 3x + 6$
  \item<4-> $f(x) + g(x) = 3x^3 + 6x^2 + 3x + 4$
  \item<5-> $f(x) - g(x) = 3x^3 + 2x^2 + 4x + 6$
  \item<6-> $f(x) \times g(x) = 6x^5 + 3x^4 + 2x^3 + 6x^2 + x + 2$
  \end{itemize}

\end{frame}

\begin{frame}
  \frametitle{Polinoma ringo kun limigita grado}
  \[
    R_q = \mathbb{Z}_q[x]/(x^n-1)
  \]

  \begin{itemize}
  \item<+-> $R_q$ estas ĉiuj polinomoj el $\mathbb{Z}_q[x]$ de grado malpli ol $n$
  \item<+-> se $h(x) = f(x) \times g(x)$ havas gradon pli ol $n-1$ ni elektas
    \[h(x) = f(x) \times g(x) \mod (x^n-1)\]
  \end{itemize}

  \vspace{0.5em}
  \uncover<+->{
    Ekzemplo $R_{41} = \mathbb{Z}_{41}[x](x^4-1)$:
    \[
      f(x) = 22x^3 + 17x^2 + 31 \; g(x) = x^3 + 19x^2 + 7x + 11
    \]
    \[
      f(x) \times g(x) = 24x^3 + 35x^2 + 35x + 39
    \]
  }
\end{frame}

\subsection*{Kyber ekzemplo}

\begin{frame}
  \frametitle{Kyber ekzemplo: krei ŝlosilojn}
  Parametroj: $q = 17, \bigl\lfloor\frac{q}{2}\bigr\rceil = 9, \; R_q = \mathbb{Z}_{17}[x](x^4 + 1)$\par
  \pause
  \vspace{0.5em}
  Privata ŝlosilo hazarde elektita
  \[ \mathbf{s} = \left( - 1x^3 - 1x^2 + 1x^1 + 0x^0, - 1x^3 +0x^2 - 1x^1 + 0x^0 \right) \]
  \pause
  Plia datumo hazarde elektita
  \[
    \mathbf{A} =
    \begin{pmatrix}
      6x^3 + 16x^2 + 16x + 11 & 9x^3 + 4x^3 + 6x + 3 \\
      5x^3 + 3x^2 + 10x +1    & 6x^3 + x^2 + 9x + 15
    \end{pmatrix}
  \]
  \[
    \mathbf{e} = \left( x^2, x^2 - x \right)
  \]
  \[
    \mathbf{t} = \mathbf{A}\mathbf{s} + \mathbf{e}
  \]
  \[
    \mathbf{t} = \left( 16x^3 + 15x^2 + 7, 10x^3 + 12x^2 + 11x + 6 \right)
  \]
  \pause
  La publika ŝlosilo konsistas el $(\mathbf{t}, \mathbf{A})$.
\end{frame}

\begin{frame}
  \frametitle{Kyber ekzemplo: ĉifro}
  Ni elektas hazarde:
  \[
    \mathbf{r} = \left( -x^3 + x^2, x^3 + x^2 - 1 \right), \; \mathbf{e_1} = \left( x^2 + x, x^2 \right), \; e_2 = -x^3 - x^2
  \]
  \pause
  Nia mesaĝo estas $11 = (1011)_2$. Kiel polinomo:
  \[m_b = x^3 + x + 1\]
  \pause
  Ni grandigas la polinomon per multobligo kun $\bigl\lfloor\frac{q}{2}\bigr\rceil$:
  \[m = \Bigl\lfloor\frac{q}{2}\Bigr\rceil m_b = 9x^3 + 9x + 9\]
  \pause
  Ni ĉifras:
  \[
    \mathbf{u} = \mathbf{A}^T\mathbf{r} + \mathbf{e_1} \;\text{kaj}\; v = \mathbf{t}^T\mathbf{r} + e_2 + m
  \]
  \pause
  Nia ĉifrita mesaĝo konsistas el
  \[
    \begin{split}
      \mathbf{u} & = (11x^3 + 11x^2 + 10x + 3, 4x^3 + 4x^2 + 13x + 11) \\
      v & = 7x^3 + 6x^2 + 8x + 15
    \end{split}
  \]
\end{frame}

\begin{frame}
  \frametitle{Kyber ekzemplo: malĉifro}
  Ni malĉifras la mesaĝon ricevante mesaĝon kun eraroj:
  \[m_e = v - \mathbf{s}^T\mathbf{u} = 7x^3 + 14x^2 + 7x + 5\]

  \pause
  \begin{columns}
    \begin{column}{0.4\textwidth}
      \begin{tikzpicture}
        \draw (0, 0) circle (5em);
        \foreach \n in {0,1,...,16} {%
          \pgfmathparse{-360 * \n / 17 + 90}\edef\a{\pgfmathresult}
          \draw [fill] (\a:5em) circle(2pt);
          \node at (\a:4.3em) {\tiny\n};
        }
        \draw (0:5em) -- (180:5em);

        \node (zero) at (90:2em) {$0$};
        \node (nine) at (270:2em) {$\bigl\lfloor \frac{q}{2} \bigr\rceil = 9$};
      \end{tikzpicture}
    \end{column}

    \begin{column}{0.6\textwidth}
      \[
        \begin{split}
          7 & \longrightarrow 9 \\
          14 & \longrightarrow 0 \\
          7 & \longrightarrow 9 \\
          5 & \longrightarrow 9
        \end{split}
      \]
      \pause
      \[\Longrightarrow m = 9x^3 + 9x + 9\]
      \pause
      \[\Longrightarrow m_b = x^3 + x + 1 \longrightarrow (1011)_2 = 11\]
    \end{column}
  \end{columns}
\end{frame}

\begin{frame}
  \frametitle{Konkludo}
  Kion ni lernis:
  \begin{itemize}
  \item Ĉifrado estas matematiko – pli ekzakte – nombra teorio
  \item Kvantumaj komputiloj ne plu estas matematika modelo
  \item Ekzistas jam ĉifrosistemoj neatakeblaj per kvantumaj komputiloj
  \item<2-> Vivu la matematiko
  \end{itemize}
\end{frame}

\begin{frame}
  \centering Dankon pro la atento
\end{frame}
\end{document}

% Local Variables:
% jinx-languages: "eo"
% TeX-master: t
% End:
